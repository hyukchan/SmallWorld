\documentclass[a4paper]{article}

\usepackage{amsmath}
\usepackage{amsfonts}
\usepackage{amssymb}
\usepackage{graphicx}
\usepackage[utf8]{inputenc}
\usepackage[T1]{fontenc}
\usepackage[francais]{babel}
\usepackage{float}
\usepackage[colorinlistoftodos]{todonotes}


\title{
\textsc{Jeu SmallWorld\\
\LARGE Rapport de conception}
}

\author
{
	Hyuk-Chan {\sc Kwon}\\
    Florent {\sc Mallard}\\
}
\date{\today}

\begin{document}
\maketitle
\tableofcontents

\section*{Introduction}
Le projet de Programmation et de Mod�lisation Orient�es Objet se porte sur la r�alisation d'un jeu inspir� de SmallWorld. Il s'agit d'un jeu de strat�gie � deux joueurs dans lequel chacun dirige un peuple. Les unit�s des peuples bougent sur les cases de la carte afin de les conqu�rir, et combattent les unit�s ennemies. Le but est de contr�ler plus de cases que son adversaire.\\
Nous allons aborder les th�mes principaux de la phase d'analyse et de conception.
Nous y expliquerons nos choix de conception  et de mod�lisation de notre jeu. Nous commencerons par effectuer un rappel des r�gles afin de distinguer les actions possibles.
Les �tapes d'analyse ainsi que les diff�rents diagrammes (classe, s�quence, cas d'utilisation) figurent �galement dans ce rapport.
Nous expliquerons �galement notre utilisation des patrons de conception.

